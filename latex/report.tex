\documentclass[12pt, a4paper, oneside]{ctexart}
\usepackage{amsmath, graphics, diagbox, float}
\graphicspath{{./figures/}}
\usepackage[left = 2.5cm, right = 2.5cm, top = 2cm, bottom = 2cm]{geometry}
\pagestyle{plain}

\ctexset{
    % 修改 section。
    section={   
        name={,、},
        number={\chinese{section}},
        format=\heiti\raggedright\zihao{4}, % 设置 section 标题为黑体、右对齐、4号字
        aftername=\hspace{0pt},
        beforeskip=1ex,
        afterskip=1ex
    },
    % 修改 subsection。
    subsection={   
        name={,、},
        number={\arabic{subsection}},
        format=\heiti\zihao{-4}, % 设置 subsection 标题为黑体、小4号字
        aftername=\hspace{0pt},
        beforeskip=1ex,
        afterskip=1ex
    }
}

\begin{document}

\begin{center}
    \Large\textbf{标题}
\end{center}

{\centering
    \textbf{姓名:}李宁 \quad 
    \textbf{学号:}PB21111715 \quad 
    \textbf{班级:}计科三班 \quad 
    \textbf{日期:} \today
}

\section{实验名称}

\section{实验目的}
\begin{enumerate}
    \item 
    \item 
    \item 
\end{enumerate}

\section{实验原理}

\section{实验仪器}
% \begin{figure}[H]
% \centering
% \includegraphics{1.png} 
% \qquad
% \includegraphics{2.png}
% \caption{图片标题}
% \end{figure} 

\section{实验步骤}
\begin{enumerate}
    \item[(1)] 
    \item[(2)] 
    \item[(3)] 
    \item[(4)] 
    \item[(5)] 
\end{enumerate}

\section{数据处理}

% \begin{table}[H]
%     \caption{表格标题}
%     \centering
%     \setlength{\tabcolsep}{5mm}{
%     \begin{tabular}{|l|c|c|c|c|c|c|c|c|c|}
%     \hline
%     \textbf{}(cm) &  &  &  &  &  &  &  &  & \\ \hline
%     \textbf{}  &  &  &  &  &  &  &  &  & \\ \hline
%     \textbf{}  &  &  &  &  &  &  &  &  & \\ \hline
%     \end{tabular}}
% \end{table}

\end{document}
