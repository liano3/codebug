\documentclass[12pt, a4paper, oneside]{ctexart}
\usepackage{amsmath, amsthm, amssymb, appendix, bm, graphicx, hyperref, mathrsfs}
\graphicspath{{./pictures/}}
\usepackage{listings, xcolor, fontspec}
\setmonofont{Consolas}
\usepackage[left = 2.5cm, right = 2.5cm, top = 2cm, bottom = 2cm]{geometry}
\linespread{1.5}
\hypersetup{
	colorlinks=true,
	linkcolor=cyan,
	filecolor=blue,      
	urlcolor=red,
	citecolor=green,
}
\lstset{
    columns=fixed,       
    numbers=left,                                        % 在左侧显示行号
    numberstyle=\tiny\color{gray},                       % 设定行号格式
    frame=none,
    backgroundcolor=\color[RGB]{245,245,244},            % 设定背景颜色
    basicstyle=\small\fontspec{Consolas},
    keywordstyle=\color[RGB]{40,40,255},                 % 设定关键字颜色
    numberstyle=\footnotesize\color{darkgray},           
    commentstyle=\it\color[RGB]{0,96,96},                % 设置代码注释的格式
    stringstyle=\rmfamily\slshape\color[RGB]{128,0,0},   % 设置字符串格式
    showstringspaces=false,                              % 不显示字符串中的空格
    language=c++,                                        % 设置语言
}
\newtheorem{theorem}{定理}[section]
\newtheorem{definition}[theorem]{定义}
\newtheorem{lemma}[theorem]{引理}
\newtheorem{corollary}[theorem]{推论}
\newtheorem{example}[theorem]{例}
\newtheorem{proposition}[theorem]{命题}

\begin{document}

\thispagestyle{empty}

\vspace*{2cm}

\begin{figure}
    \centering
    \includegraphics[width = 13cm]{logo.pdf}
\end{figure}

\begin{center}
    \Huge\textbf{报告或论文模板}
\end{center}

\vspace*{5cm}

\begin{table}[htpb]
    \centering
    \large
    \begin{tabular}{ll}
        \textbf{姓名:} & 李宁 \\
        \textbf{学号:} & PB21111715 \\
        \textbf{班级:} & 计科三班 \\
        \textbf{日期:} & \today \\
    \end{tabular}    
\end{table}

\newpage
\setcounter{page}{0}
\thispagestyle{empty}
\vspace*{1cm}
\renewcommand{\abstractname}{\Large\textbf{摘要}}
\begin{abstract}
    \vspace*{0.5cm}
    这里是摘要. 
    \vspace*{0.5cm}
    \par\textbf{关键词:}这里是关键词; 这里是关键词. 
\end{abstract}

\newpage
\pagenumbering{Roman}
\setcounter{page}{1}
\tableofcontents

\newpage
\setcounter{page}{1}
\pagenumbering{arabic}

\section{一级标题}

\subsection{二级标题}

这里是正文.
\begin{lstlisting}
#include<stdio.h>
int main()  //主函数
{
    printf("hello, world!");
}
return 0;
\end{lstlisting}

\subsection{二级标题}

这里是正文.
% \begin{figure}[htbp]
%     \centering
%     \includegraphics[width=8cm]{test.jpg}
%     \caption{图片标题}
% \end{figure} 

\begin{enumerate}
    \item[(1)] 这是第一点; 
    \item[(2)] 这是第二点;
    \item[(3)] 这是第三点. 
\end{enumerate}

\begin{itemize}
    \item 这是一点; 
    \item 这是一点;
    \item 这是一点. 
\end{itemize}

\newpage

\begin{table}[htbp]
    \caption{这是一个表格}
    \centering
    \begin{tabular}{|c|c|c|c|c|}
    \hline
    表头 & 一 & 二 & 三 \\ \hline
    一  & 6 & 6 & 6 \\ \hline
    二  & 6 & 6 & 6 \\ \hline
    \end{tabular}
\end{table}

\newpage

\begin{thebibliography}{99}
    \bibitem{a}作者. \emph{文献}[M]. 地点:出版社,年份.
    \bibitem{b}作者. \emph{文献}[M]. 地点:出版社,年份.
\end{thebibliography}

\newpage

\begin{appendices}
    \renewcommand{\thesection}{\Alph{section}}
    \section{附录标题}
        这里是附录. 
\end{appendices}

\end{document}